\PassOptionsToPackage{unicode=true}{hyperref} % options for packages loaded elsewhere
\PassOptionsToPackage{hyphens}{url}
%
\documentclass[]{article}
\usepackage{lmodern}
\usepackage{amssymb,amsmath}
\usepackage{ifxetex,ifluatex}
\usepackage{fixltx2e} % provides \textsubscript
\ifnum 0\ifxetex 1\fi\ifluatex 1\fi=0 % if pdftex
  \usepackage[T1]{fontenc}
  \usepackage[utf8]{inputenc}
  \usepackage{textcomp} % provides euro and other symbols
\else % if luatex or xelatex
  \usepackage{unicode-math}
  \defaultfontfeatures{Ligatures=TeX,Scale=MatchLowercase}
\fi
% use upquote if available, for straight quotes in verbatim environments
\IfFileExists{upquote.sty}{\usepackage{upquote}}{}
% use microtype if available
\IfFileExists{microtype.sty}{%
\usepackage[]{microtype}
\UseMicrotypeSet[protrusion]{basicmath} % disable protrusion for tt fonts
}{}
\IfFileExists{parskip.sty}{%
\usepackage{parskip}
}{% else
\setlength{\parindent}{0pt}
\setlength{\parskip}{6pt plus 2pt minus 1pt}
}
\usepackage{hyperref}
\hypersetup{
            pdftitle={Hello World},
            pdfauthor={Me},
            pdfborder={0 0 0},
            breaklinks=true}
\urlstyle{same}  % don't use monospace font for urls
\usepackage[margin=1in]{geometry}
\usepackage{graphicx,grffile}
\makeatletter
\def\maxwidth{\ifdim\Gin@nat@width>\linewidth\linewidth\else\Gin@nat@width\fi}
\def\maxheight{\ifdim\Gin@nat@height>\textheight\textheight\else\Gin@nat@height\fi}
\makeatother
% Scale images if necessary, so that they will not overflow the page
% margins by default, and it is still possible to overwrite the defaults
% using explicit options in \includegraphics[width, height, ...]{}
\setkeys{Gin}{width=\maxwidth,height=\maxheight,keepaspectratio}
\setlength{\emergencystretch}{3em}  % prevent overfull lines
\providecommand{\tightlist}{%
  \setlength{\itemsep}{0pt}\setlength{\parskip}{0pt}}
\setcounter{secnumdepth}{0}
% Redefines (sub)paragraphs to behave more like sections
\ifx\paragraph\undefined\else
\let\oldparagraph\paragraph
\renewcommand{\paragraph}[1]{\oldparagraph{#1}\mbox{}}
\fi
\ifx\subparagraph\undefined\else
\let\oldsubparagraph\subparagraph
\renewcommand{\subparagraph}[1]{\oldsubparagraph{#1}\mbox{}}
\fi

% set default figure placement to htbp
\makeatletter
\def\fps@figure{htbp}
\makeatother

\usepackage{tikz}
\usepackage{pgfplots}
\usepackage{pgf,tikz,pgfplots}
\pgfplotsset{compat=1.15}
\usepackage{mathrsfs}
\usetikzlibrary{arrows}
\usetikzlibrary{patterns}

\title{Hello World}
\author{Me}
\date{February 24, 2020}

\begin{document}
\maketitle

\hypertarget{tikz-picture}{%
\subsection{TikZ picture}\label{tikz-picture}}

\begin{itemize}
\tightlist
\item
  Here is a TikZ picutre
\end{itemize}

\definecolor{zzffzz}{rgb}{0.6,1,0.6}
\definecolor{ffcctt}{rgb}{1,0.8,0.2}
\definecolor{yqyqdz}{rgb}{0.5019607843137255,0.5019607843137255,0.8509803921568627}
\begin{tikzpicture}
  [node distance=.8cm,
  start chain=going below,]
     \node[punktchain, join] (intro) {Introduktion};
     \node[punktchain, join] (probf)      {Problemformulering};
     \node[punktchain, join] (investeringer)      {Investeringsteori};
     \node[punktchain, join] (perfekt) {Det perfekte kapitalmarked};
     \node[punktchain, join, ] (emperi) {Emperi};
      \node (asym) [punktchain ]  {Asymmetrisk information};
      \begin{scope}[start branch=venstre,
        %We need to redefine the join-style to have the -> turn out right
        every join/.style={->, thick, shorten <=1pt}, ]
        \node[punktchain, on chain=going left, join=by {<-}]
            (risiko) {Risiko og gamble};
      \end{scope}
      \begin{scope}[start branch=hoejre,]
      \node (finans) [punktchain, on chain=going right] {Det finansielle system};
    \end{scope}
  \node[punktchain, join,] (disk) {Det imperfekte finansielle marked};
  \node[punktchain, join,] (makro) {Investeringsmæssige konsekvenser};
  \node[punktchain, join] (konk) {Konklusion};
  % Now that we have finished the main figure let us add some "after-drawings"
  %% First, let us connect (finans) with (disk). We want it to have
  %% square corners.
  \draw[|-,-|,->, thick,] (finans.south) |-+(0,-1em)-| (disk.north);
  % Now, let us add some braches. 
  %% No. 1
  \draw[tuborg] let
    \p1=(risiko.west), \p2=(finans.east) in
    ($(\x1,\y1+2.5em)$) -- ($(\x2,\y2+2.5em)$) node[above, midway]  {Teori};
  %% No. 2
  \draw[tuborg, decoration={brace}] let \p1=(disk.north), \p2=(makro.south) in
    ($(2, \y1)$) -- ($(2, \y2)$) node[tubnode] {Analyse};
  %% No. 3
  \draw[tuborg, decoration={brace}] let \p1=(perfekt.north), \p2=(emperi.south) in
    ($(2, \y1)$) -- ($(2, \y2)$) node[tubnode] {Problemfelt};
  \end{tikzpicture}

\end{document}
